\begin{abstract}
Selfish node detection offers an effective means
to mitigate the routing performance degradation
caused by selfish behaviors in Opportunistic Networks (OppNets),
but leads to the extra network overload and computation cost.
%due to the detection behaviors.
Most existing effort in the literature focuses on 
exploring the detection methods
based on the traffic analysis or the cooperations among nodes.
%However, the detection brings about
%the detection cost, e.g. energy, bandwidth, revenue,
%as well as the decrement of selfish nodes.
In this paper, we investigate the state transition of nodes
in the message dissemination without detection.
Specifically, the Ordinary Differential Equation (ODE) is constructed 
to approximatively model the periodic detection
with complete detection requirement.
Then we propose the optimal detection solution
with the Pontryagin's maximum principle, 
and mathematically deduce the right detection time 
%the moment of triggering detection on/off
during the message lifetime.
The model soundness is verified statistically
and the analysis accuracy is evaluated via extensive simulations.
The experiments also show that our solution can
achieve the tradeoff between the reward and the detection cost.
%reduce the combinational cost of the wasted reward and the detection.
%based on RWP trace and Shanghai taxi trace.
%Extensive simulations show that
%our scheme outperforms
%the benchmarks, including SDBG, Li and MDS,
%in terms of the delivery ratio in various scenarios.
\end{abstract}
