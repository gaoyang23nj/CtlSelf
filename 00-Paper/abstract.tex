\begin{abstract}
Selfish detection in the opportunistic networks offers an effective means
to mitigate the routing performance degradation,
but faces many challenges from the balance between 
the cost and the effect of detection behaviors.
%Blackhole detection in the opportunistic networks
%offers an effective means
%to mitigate the routing performance degradation,
%but faces many challenges from corrupted nodes
%due to their collusion behaviors.
%Most existing effort in the literature focuses on
%the blackhole feature extraction from the message exchange.
%However, the decay effect of features
%and the forged features from the corrupted node,
%which acts as the rational node in performing message exchange,
%degrade the performance of the detection.
%In this paper, we investigate the evidence construction,
%i.e., the direct and indirect evidence with the statistical parameters
%in message exchange.
%Specifically, we construct behavior classifiers
%to distinguish the blackhole behaviors from rational ones
%and design the collusion filtering strategy
%to improve the detection accuracy by
%separating corrupted nodes from rational ones, respectively,
%laying a behavior identification foundation.
%The Contact Evidence-driven Blackhole Detection
%based on machine learning (CEBD)
%is proposed to improve the routing performance.
%The soundness of the proposed scheme is verified statistically
%and the detection accuracy is evaluated
%based on RWP trace and Shanghai taxi trace.
%Extensive simulations show that
%our scheme outperforms
%the benchmarks, including SDBG, Li and MDS,
%in terms of the delivery ratio in various scenarios.
\end{abstract}
