\section{Introduction}
\label{sec:intro}
Exploiting mobile nodes to transmit message in
OppNets has been attracting
increasing research attentions
\cite{DBLP:conf/sigcomm/SouzaMSMCC16,
DBLP:conf/infocom/LuSP16,
DBLP:conf/mobicom/RadenkovicH17,
DBLP:journals/comsur/JedariXN18,
DBLP:journals/tmc/LoretiB20}.
With a widespread use and availability of
mobile communication devices, numerous applications
emerge based on message transmission in OppNets,
especially in mobile social networks (OMSNs).
\cite{DBLP:journals/tsc/KhalidKKZ14} builded a
cloud-based recommendation system OmniSuggest 
based on OMSNs. In this project, user activities,
mobility patterns are tracked to provide 
The opportunistic computing technology utilize
the shared resources of OMSNs to
provide optimal venue recommendations.
More examples can be found in mobile data offloading,
such as~\cite{DBLP:journals/tmc/HanHKMSS12,
DBLP:journals/tmc/LiQJHW014}.

Traditional message dissemination approaches
heavily rely on voluntary cooperation between
mobile nodes, which excessively consumes the
limited energy supply and lead to massive useless
message copies due to the selfish behaviour.
Adopting selfish node detection mechanisms
to avoid selfish node involved in data forwarding,
reduces and balances the communication loads
of nodes (and thus their energy consumptions).
However, selfish node detection leads to network
management cost due to the detecting expense, and
introduce extra detection traffic,
degrading the overall performance of OppNets.

Much research effort on selfish node detection
exists in the literature.




%With a widespread use and availability of
%mobile communication devices, numerous applications
%emerge based on message transmission in OppNets,
%especially in mobile social networks (OMSNs).
%The opportunistic computing technology utilize
%the shared resources of OMSNs to
%provide a platform for the execution
%of distributed computing tasks,

The main contributions are as follows:

\begin{itemize}
\item {we formulate the ordinary differential equation model (ODE)
to capture and analytically evaluate the state transition of nodes
in OppNets without detection and with complete detection.}
\item {we propose an optimal solution of selfish node detection
based on the Pontryagin's maximum principle
to achieve the tradeoff between the detection cost
and the reward of selfish behaviors.}
\item {we conduct experiments to evaluate
the effectiveness of the proposed model
and the optimal selfish detection solution
in terms of the total cost, the wasted reward and the node state transition.}
\end{itemize}

The rest of this paper is organized as follows.
The literature is reviewed in Section~\ref{sec:related}.
We formulate the problem in Section~\ref{sec:preli}.
The change of network state without detection and with fully detection
is investigated in Section~\ref{sec:ode_model}.
The optimal solution of the selfish detection in OppNets
is presented in Section~\ref{sec:opt_detect},
and evaluated in Section~\ref{sec:pe}.
The paper concludes in Section~\ref{sec:conclude}.