\section{Introduction}
\label{sec:intro}
Exploiting mobile nodes to transmit message in
opportunistic networks(OppNets) has been attracting
increasing research attentions
\cite{DBLP:conf/sigcomm/SouzaMSMCC16,
DBLP:conf/infocom/LuSP16,
DBLP:conf/mobicom/RadenkovicH17,
DBLP:conf/infocom/SakaiSK19,
DBLP:journals/comsur/JedariXN18,
DBLP:journals/tmc/LoretiB20}.
With a widespread use and availability of
mobile communication devices, numerous applications
emerge based on message transmission in OppNets, especially in mobile social networks(OMSNs).
Opportunistic computing utilize the shared resources
in  OMSNs to provide a platform for the execution of
distributed computing tasks,



The main contributions are as follows:

\begin{itemize}
\item {we formulate the ordinary differential equation model (ODE)
to capture and analytically evaluate the change of network state
in OppNets without detection and with fully detection.}
%\item {we propose a ODE model with continuous strategy
%to model the periodic detection in the case of fully detection.}
\item {we propose an optimal solution of selfish node detection
based on the Pontryagin's maximum principle
to achieve the balance between the detection cost
and the wasted reward of selfish behaviors.
}
\item {we conduct experiments to evaluate
the effectiveness of the proposed model
and the optimal selfish detection solution in terms of the total cost,
the wasted reward and the change of network state.
}
\end{itemize}

The rest of this paper is organized as follows.
The literature is reviewed in Section~\ref{sec:related}.
We formulate the problem in Section~\ref{sec:preli}.
The change of network state without detection and with fully detection
is investigated in Section~\ref{sec:ode_model}.
The optimal solution of the selfish detection in OppNets
is presented in Section~\ref{sec:opt_detect},
and evaluated in Section~\ref{sec:pe}.
The paper concludes in Section~\ref{sec:conclude}. 