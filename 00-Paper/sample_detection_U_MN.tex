\section{Optimal Detection}
\label{opt_detect}
\subsection{Problem Formulation}
Assume that the detection can be conducted.
The detection rate is $U(t)$, $0 \le U(t) \le U_{m}$.
$U_{m}$ is the limitation of the detection rate, which is the constraint from the hardware and the time sequences.
We also use $\dot{U}$ to denote $U(t)$.
Then, the ODEs can be reformed as
\begin{small}
\begin{equation}
\label{eq:SIM_t}
%\nonumber
\begin{aligned}
\dot{I} &= \beta (N-I) - \rho I, \\
\dot{M} &= \rho I  - \beta M - \frac{M}{N} U,\\
\dot{S} &= - \beta (N-I-M) + \frac{M}{N} U.
\end{aligned}
\end{equation}
\end{small}
Meanwhile,
\begin{small}
\begin{equation}
\label{eq:SIM_0}
%\nonumber
\begin{aligned}
I(0)=0,\\
M(0)=0,\\
S(0)=N.
\end{aligned}
\end{equation}
\end{small}

Thus $I(t)$ is the same with that in the situation without detection, which is
\begin{small}
\begin{equation}
%\nonumber
\label{eq:I}
\begin{aligned}
I(t) = \frac{ \beta N }{ \beta + \rho } - \frac{ \beta N }{ \beta + \rho } e^{-(\beta + \rho)t}.
\end{aligned}
\end{equation}
\end{small}

Considering that the detection is also the cost,
the object function will be
\begin{small}
\begin{equation}
\nonumber
\begin{aligned}
J &= \int_{0}^{T} (1-\alpha) M + \alpha U dt.
\end{aligned}
\end{equation}
\end{small}
Here $\alpha$ is the weight, which can control the importance
between the cost of selfish relay nodes
and detections.
Thus $0 < \alpha < 1$.
Similar with the previous section,
$I(t)$ and $M(t)$ is the state variable.
$U(t)$ is the controllable variable, $0 \le U(t)\ \le U_{m}$.

\subsection{Optimal Control by Pontryagin's Maximal Principle}
Now we utilize the Pontryagin's maximal principle to find the optimal $U(t)$, which will minimize the total cost.
First, the Hamilton function is
\begin{small}
\begin{equation}
\nonumber
\begin{aligned}
H =& (1-\alpha) M + \alpha U + \lambda_{1} (\beta (N-I) - \rho I) \\
& + \lambda_{2} (\rho I  - \beta M - \frac{M}{N} U) \\
=& (1-\alpha) M + \alpha U + \lambda_{1} (\beta (N-I) - \rho I) + \lambda_{2} \rho I\\
& - \beta \lambda_{2} M - \lambda_{2} \frac{1}{N} U M \\
=& (1-\alpha) M + \lambda_{1} (\beta (N-I) - \rho I) \\
& + \lambda_{2} (\rho I  - \beta M) + ( \alpha - \lambda_{2} \frac{M}{N}) U.
\end{aligned}
\end{equation}
\end{small}
Note that $\lambda_{1}$ and $\lambda_{2}$ denote $\lambda_{1}(t)$ and $\lambda_{2}(t)$, respectively.
Without the final constraint, the terminal condition is $\lambda_{2}(T) = 0$ and $\lambda_{3}(T) = 0$.
Then the adjoint function is
\begin{small}
\begin{equation}
\nonumber
\begin{aligned}
\dot{\lambda_{2}} &= - \frac{ \partial H}{ \partial M} = \lambda_{2} (\beta + \frac{U}{N} ) - (1-\alpha).
\end{aligned}
\end{equation}
\end{small}

Thus
\begin{small}
\begin{equation}
\label{eq:opt_U}
%\nonumber
U^{*}(t) =
\left\{
\begin{aligned}
&0,      & \text{if }  \alpha - \lambda_{2} \frac{M}{N} \ge 0 \\
&U_{m},        & \text{if } \alpha - \lambda_{2} \frac{M}{N} < 0
\end{aligned}
\right.
\end{equation}
\end{small}

In summary, we have the ODE functions $\dot{M}$, $\dot{\lambda_{2}}$,
the initial condition $M(0)=0$ and the boundary condition $\lambda_{2}(T)=0$,
Thus the problem is to solve a BVP problem,
which is
\begin{small}
\begin{equation}
%\nonumber
\label{eq:bvp}
\begin{aligned}
\dot{M} &= \rho I  - \beta M - \frac{M}{N} U,\\
\dot{\lambda_{2}} &= - \frac{ \partial H}{ \partial M} = \lambda_{2} (\beta + \frac{U}{N} ) - (1-\alpha),\\
M(0) &= 0,\\
\lambda_{2}(T) &= 0.
\end{aligned}
\end{equation}
\end{small}
We can solve the BVP problem with the shooting method by the bvpSolve package of R.
%footnote
Then we analyze the properties of the optimal control variable.
\begin{lem}
There exists a unique solution of (\ref{eq:bvp}).
\end{lem}
\begin{proof}
(\ref{eq:bvp}) comforts to the Lipschitz condition.
Then the solution exists and the solution is unique. ... (ODE).
\end{proof}

\begin{lem}At the beginning and the end of the whole duration,
the optimal control stop the selfish detection,
which is $U(0)=U(T)=0$.
\end{lem}

\begin{proof}
At the beginning of the duration, $M(0)=0$,
which is the initial condition of \ref{eq:bvp}.
Then $\alpha - \lambda_{2}(0) \frac{M(0)}{N} = \alpha > 0$.
Following (\ref{eq:opt_U}), the optimal $U(0)=0$.

At the end of the duration, $\lambda_{2}(T)=0$,
which is the boundary condition of \ref{eq:bvp}.
Then $\alpha - \lambda_{2}(T) \frac{M(T)}{N} = \alpha > 0$.
Based on (\ref{eq:opt_U}), the optimal $U(T)=0$.
\end{proof}

Based on the differential function $\dot{I}$,
the equilibrium point of $I$ can be obtained from $\dot{I}=0$,
which is $I^{*}=\frac{\beta N}{\beta+\rho}$.
When $I(t)<I^{*}$, $I(t)$ will increase with $t$ and approach to $\frac{\beta N}{\beta+\rho}$.
Meanwhile, in this paper $I(0)=0$ at the beginning of time.

Based on the differential function $\dot{M}$,
the equilibrium point is obtained from $\dot{M}=0$,
which is $M^{*}=\frac{\rho I}{\beta+\frac{1}{N}U}$.
In the situation without detection,
the equilibrium point is $M^{*}=\frac{\rho I^{*}}{\beta}=\frac{\rho N}{\beta + \rho}$.
In the situation with full detection,
the equilibrium point is $M^{*}=\frac{\rho I^{*}}{\beta + \frac{1}{N} U_{m}}=\frac{\rho}{\beta + \frac{1}{N} U_{m}} \frac{\beta N}{\beta+\rho}$.

Since $\alpha$ is the weight of detecting the selfish nodes,
we can assume that if $\alpha$ is enough high, the detection will not perform according to the optimal control strategy.
\begin{lem}
If $\alpha \ge \alpha_{th}$, the optimal control let the detection stop in the whole duration,
namely $U(t)=0$, $0 \le t \le T$.
\end{lem}

\begin{proof}
Assume that $\rho$, $N$, $\beta$ is given.
Let $W(t) = \lambda_{2}(t)M(t)$.
\begin{small}
\begin{equation}
%\nonumber
\label{eq:W_diff}
\begin{aligned}
W^{'}(t) =& M^{'}(t) \lambda_{2}(t) + M(t) \lambda_{2}^{'}(t)\\
=& (\rho I(t)  - \beta M(t) - \frac{M(t)}{N} U(t) )\lambda_{2}(t) \\
&+ M(t) (\lambda_{2}(t) (\beta + \frac{U(t)}{N} ) - (1-\alpha))\\
=& \rho \lambda_{2}(t) I(t) - (1-\alpha)M(t).
\end{aligned}
\end{equation}
\end{small}
Since $M(0)=0$ and $\lambda_{2}(T)=0$,
$W(0)=W(T)=0<\alpha N$.

Now we focus on the poles of $W(t)$, namely $t^{*}$,
where $W^{'}(t^{*})=\rho \lambda_{2}(t^{*}) I(t^{*}) - (1-\alpha)M(t^{*})=0$.
Then $M(t^{*}) = \frac{\rho \lambda_{2}(t^{*}) I(t^{*})}{1-\alpha}$.
\begin{small}
\begin{equation}
%\nonumber
\label{eq:W_t_star}
\begin{aligned}
W(t^{*}) &= \lambda_{2}(t^{*}) M(t^{*}) = \frac{\rho I(t^{*}) \lambda_{2}(t^{*})^2}{1-\alpha}.
\end{aligned}
\end{equation}
\end{small}

According to $\dot{\lambda_{2}}$ in (\ref{eq:bvp}),
the equilibrium point of $\lambda_{2}$ is that $\lambda_{2}^{*} = \frac{1-\alpha}{\beta+\frac{U}{N}}$.
Since $0 \le U \le U_{m}$, $0 < \frac{1-\alpha}{\beta+\frac{U_{m}}{N}} \le \lambda_{2}^{*} \le \frac{1-\alpha}{\beta}$.
Note $\lambda_{2}(T)=0$.
Based on the phase line in ODE for $\dot{\lambda_{2}}$,
$\lambda_{2}(t)$ decreases with $t$ when $\lambda_{2}(t) < \lambda_{2}^{*}$.
Conversely, $\lambda_{2}(t)$ increases with $t$ when $\lambda_{2}(t) > \lambda_{2}^{*}$.
Thus $0 \le \lambda_{2}(t) \le \lambda_{2}^{*} \le \frac{1-\alpha}{\beta}$ when $0 \le t \le T$.
Additionally, $0 \le I(t) \le \frac{\beta N}{\beta + \rho}$.
From (\ref{eq:W_t_star}), we can derive that the upper boundary of $W(t)$, $W_{up}$, which is
\begin{small}
\begin{equation}
\nonumber
\begin{aligned}
W(t) \le W(t^{*}) \le \frac{\rho}{1-\alpha} \frac{\beta N}{\beta + \rho} (\frac{1-\alpha}{\beta})^2 = \frac{\rho N (1-\alpha)}{\beta(\beta+\rho)} = W_{up}.
\end{aligned}
\end{equation}
\end{small}

Assume that $\alpha$ can satisfy that $W_{up} \le \alpha N$,
which means that $\alpha \ge \frac{\rho}{\beta(\beta+\rho)+\rho} = \alpha_{th}$.
Then $W(t) \le \alpha N$, when $0 \le t \le T$.
Therefore the optimal control $U^{*}(t) \equiv 0$, when $0 \le t \le T$.
\end{proof}

