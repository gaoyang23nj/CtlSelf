\section{Optimal Detection}
\label{opt_detect}
\subsection{Problem Formulation}
Assume that the detection can be conducted.
The detection rate is $U(t)$, $0 \le U(t) \le U_{m}$.
$U_{m}$ is the limitation of the detection rate, which is the constraint from the hardware and the time sequences.
We also use $\dot{U}$ to denote $U(t)$.
Then, the ODEs can be reformed as
\begin{small}
\begin{equation}
\label{eq:SIM_t}
%\nonumber
\begin{aligned}
\dot{I} &= \beta (N-I) - \rho I, \\
\dot{M} &= \rho I  - \beta M - \frac{M}{N} U,\\
\dot{S} &= - \beta (N-I-M) + \frac{M}{N} U.
\end{aligned}
\end{equation}
\end{small}
Meanwhile, 
\begin{small}
\begin{equation}
\label{eq:SIM_0}
%\nonumber
\begin{aligned}
I(0)=0,\\
M(0)=0,\\
S(0)=N.
\end{aligned}
\end{equation}
\end{small}

Thus $I(t)$ is the same with that in the situation without detection, which is
\begin{small}
\begin{equation}
%\nonumber
\label{eq:I}
\begin{aligned}
I(t) = \frac{ \beta N }{ \beta + \rho } - \frac{ \beta N }{ \beta + \rho } e^{-(\beta + \rho)t}.
\end{aligned}
\end{equation}
\end{small}

Considering that the detection is also the cost,
the object function will be
\begin{small}
\begin{equation}
\nonumber
\begin{aligned}
J &= \int_{0}^{T} (1-\alpha) M + \alpha U dt.
\end{aligned}
\end{equation}
\end{small}
Here $\alpha$ is the weight, which can control the importance 
between the cost of selfish relay nodes
and detections.
Thus $0 < \alpha < 1$.
Similar with the previous section,
$I(t)$ and $M(t)$ is the state variable.
$U(t)$ is the controllable variable, $0 \le U(t)\ \le U_{m}$.

\subsection{Optimal Control by Pontryagin's Maximal Principle}
Now we utilize the Pontryagin's maximal principle to find the optimal $U(t)$, which will minimize the total cost.
First, the Hamilton function is
\begin{small}
\begin{equation}
\nonumber
\begin{aligned}
H =& (1-\alpha) M + \alpha U + \lambda_{1} (\beta (N-I) - \rho I) \\
& + \lambda_{2} (\rho I  - \beta M - \frac{M}{N} U) \\
=& (1-\alpha) M + \alpha U + \lambda_{1} (\beta (N-I) - \rho I) + \lambda_{2} \rho I\\
& - \beta \lambda_{2} M - \lambda_{2} \frac{1}{N} U M \\
=& (1-\alpha) M + \lambda_{1} (\beta (N-I) - \rho I) \\
& + \lambda_{2} (\rho I  - \beta M) + ( \alpha - \lambda_{2} \frac{M}{N}) U.
\end{aligned}
\end{equation}
\end{small}
Note that $\lambda_{1}$ and $\lambda_{2}$ denote $\lambda_{1}(t)$ and $\lambda_{2}(t)$, respectively.
Without the final constraint, the terminal condition is $\lambda_{2}(T) = 0$ and $\lambda_{3}(T) = 0$.
The adjoint function is
\begin{small}
\begin{equation}
\nonumber
\begin{aligned}
\dot{\lambda_{2}} &= - \frac{ \partial H}{ \partial M} = \lambda_{2} (\beta + \frac{U}{N} ) - (1-\alpha).
\end{aligned}
\end{equation}
\end{small}

Thus
\begin{small}
\begin{equation}
\label{eq:opt_U}
%\nonumber
U^{*}(t) =
\left\{
\begin{aligned}
&0,      & \text{if }  \alpha - \lambda_{2} \frac{M}{N} \ge 0 \\
&U_{m},        & \text{if } \alpha - \lambda_{2} \frac{M}{N} < 0
\end{aligned}
\right.
\end{equation}
\end{small}

In summary, we have the ODE functions $\dot{M}$, $\dot{\lambda_{2}}$,
the initial condition $M(0)=0$ and the boundary condition $\lambda_{2}(T)=0$,
Thus the problem is to solve a BVP problem,
which is
\begin{small}
\begin{equation}
%\nonumber
\label{eq:bvp}
\begin{aligned}
\dot{M} &= \rho I  - \beta M - \frac{M}{N} U,\\
\dot{\lambda_{2}} &= - \frac{ \partial H}{ \partial M} = \lambda_{2} (\beta + \frac{U}{N} ) - (1-\alpha),\\
M(0) &= 0,\\
\lambda_{2}(T) &= 0.
\end{aligned}
\end{equation}
\end{small}
We can solve the BVP problem with the shooting method by the bvpSolve package of R.
%footnote
