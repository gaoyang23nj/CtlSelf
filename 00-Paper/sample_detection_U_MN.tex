\section{Optimal Detection}
\label{sec:opt_detect}
Here we propose the optimal method to
control the selfish detection rate about the relay nodes.
with the Pontryagin's maximum principle.
Then we derive the properties of 
the optimal control variable $U^{*}(t)$.
The whole process of our method is also presented.

\subsection{Problem Formulation}
Assume that the detection can be conducted.
The detection rate is $U(t)$, $0 \le U(t) \le U_{m}$.
$U_{m}$ is the limitation of the detection rate, which is the constraint from the hardware and the time sequences.
Then, the ODEs can be reformulated as
\begin{equation}
\begin{small}
\label{eq:SIM_t}
%\nonumber
\begin{aligned}
\frac{\mathrm{d} I(t)}{\mathrm{d} t} &=
\lambda (N-I(t)) - \rho I(t), \\
\frac{\mathrm{d} D(t)}{\mathrm{d} t} &=
\rho I(t)  - \lambda D(t) - \frac{D(t)}{N} U(t),\\
\frac{\mathrm{d} R(t)}{\mathrm{d} t} &=
- \beta (N-I(t)-D(t)) + \frac{D(t)}{N} U(t).
\end{aligned}
\end{small}
\end{equation}
Meanwhile, we have $I(0)=D(0)=0$ and $R(0)=N$.
Thus $I(t)$ is the same with that in the situation without detection, which is
\begin{equation}
\begin{small}
%\nonumber
\label{eq:I}
\begin{aligned}
I(t) = \frac{ \lambda N }{ \lambda + \rho }(1- e^{-(\lambda + \rho)t}).
\end{aligned}
\end{small}
\end{equation}

Considering that the detection is also the cost,
the object function will be
\begin{equation}
\begin{small}
\nonumber
\begin{aligned}
J &= \int_{0}^{T} (1-\alpha) D + \alpha U dt.
\end{aligned}
\end{small}
\end{equation}
Here $\alpha$ is the weight, which can control the importance
between the cost of selfish relay nodes
and detections.
Thus $0 < \alpha < 1$.
Similar with the previous section,
$I(t)$ and $D(t)$ is the state functions.
$U(t)$ is the controllable variable, $0 \le U(t)\ \le U_{m}$.

\subsection{Optimal Control by Pontryagin's Maximum Principle}
Now we utilize the Pontryagin's maximum principle~\cite{DBLP:journals/tcss/WuDH18}
to find the optimal $U(t)$, which will minimize the total cost.
First, the Hamilton function is
\begin{equation}
\begin{small}
\nonumber
\begin{aligned}
H =& (1-\alpha) D + \alpha U + \lambda_{I} (\lambda (N-I) - \rho I) \\
& + \lambda_{D} (\rho I  - \lambda D - \frac{D}{N} U) \\
%=& (1-\alpha) M + \alpha U + \lambda_{1} (\beta (N-I) - \rho I) + \lambda_{2} \rho I\\
%& - \beta \lambda_{2} M - \lambda_{2} \frac{1}{N} U M \\
=& (1-\alpha) D + \lambda_{I} (\lambda (N-I) - \rho I) \\
& + \lambda_{D} (\rho I  - \lambda D) + ( \alpha - \lambda_{D} \frac{D}{N}) U.
\end{aligned}
\end{small}
\end{equation}
Note that $\lambda_{I}$ and $\lambda_{D}$ 
denote two co-state functions.
Without the final constraint, the terminal condition is
$\lambda_{I}(T) = 0$ and $\lambda_{D}(T) = 0$.
Then the adjoint function is
\begin{equation}
\begin{small}
\nonumber
\begin{aligned}
\dot{\lambda_{D}} = - \frac{ \partial H}{ \partial D}
= \lambda_{D} (\lambda + \frac{U}{N} ) - (1-\alpha).
\end{aligned}
\end{small}
\end{equation}

Thus
\begin{equation}
\begin{small}
\label{eq:opt_U}
%\nonumber
U^{*}(t) =
\left\{
\begin{aligned}
&0,      & \text{if }  \alpha - \lambda_{D} \frac{D}{N} \ge 0 \\
&U_{m},        & \text{if } \alpha - \lambda_{D} \frac{D}{N} < 0
\end{aligned}
\right.
\end{small}
\end{equation}

In summary, we have the ODE functions $\dot{D}$, $\dot{\lambda_{D}}$,
the initial condition $D(0)=0$ and 
the boundary condition $\lambda_{D}(T)=0$,
Thus the problem is to solve a BVP problem,
which is
\begin{equation}
\begin{small}
\label{eq:bvp}
\begin{aligned}
\dot{D} &= - (\lambda + \frac{U^{*}}{N}) D + \rho I,\\
\dot{\lambda_{D}} &= (\lambda + \frac{U^{*}}{N} ) \lambda_{D} - (1-\alpha),\\
\end{aligned}
\end{small}
\end{equation}
where $D(0) = 0$ and $\lambda_{D}(T) = 0$.
We can solve the BVP problem with the shooting method
by the bvpSolve package of R.
%footnote code link!!!
Then we analyze the properties of the optimal control $U^{*}(t)$.
\begin{algorithm}
\caption{Optimal Selfish Node Detection}
\label{alg:opt_detection}
\begin{small}
\begin{algorithmic}[1]
\REQUIRE $T_{m}$, ${U}_{m}$, $\lambda$, $\rho$, $T$
\STATE {time $t$ $\leftarrow$ $0$}
\STATE {compute the solution of (\ref{eq:bvp})
as the switch-on duration $\mathcal{T}$}
\WHILE {$t \le T$}
    \IF {contact $n_{i}$ without message}
        \STATE {replicate $m$ to $n_{i}$}
        \STATE {state of $n_{i}$ changes to state $I$}
        \STATE {$src$ record $t$ as the message replication time}
    \ENDIF
    \IF {$t$ $\in$ $\mathcal{T}$}
        \STATE {select a relay node $n_{j}$ randomly}
        \STATE {$src$ conduct the selfish node detection to $n_{j}$}
            \IF {$n_{j}$ is detected as a selfish node}
                \STATE {state of $n_{j}$ changes to state $R$}
                \STATE {$src$ record $t$ as the state switching time}
            \ENDIF
    \ENDIF
    \STATE {$t \leftarrow t + \delta t$}
\ENDWHILE
\FOR {$n_{i}$, $1 \le i \le N$}
    \STATE {pay reward to $n_{i}$ based on
    the time of staying in state $I$}
\ENDFOR
\end{algorithmic}
\end{small}
\end{algorithm}

%BVP Problem: solution exist and unique
%BVP?a��?��??��o��?����?D??������
%\begin{lem}
%There exists a unique solution of (\ref{eq:bvp}).
%\end{lem}
%\begin{proof}
%The solution space is $R: 0 \le t \le T, $
%
%(\ref{eq:bvp}) comforts to the Lipschitz condition.
%Then the solution exists and is unique.(ODE).
%
%Since $D(0) = 0$ and $\le D(t) \le N$,
%$|D(T) - D(0)| \le \frac{N}{T} |T-0|$.
%Thus $D(t)$ comforts to the Lipschitz conduction,
%and the corresponding Lipschitz parameter is $\frac{N}{T}$.
%\end{proof}
\begin{lem}\label{lem:Ut0}
At the beginning and the end of the whole duration,
the optimal control stop the selfish node detection,
which means $U(0)=U(T)=0$.
\end{lem}

\begin{proof}
At the beginning of the duration, $M(0)=0$,
which is the initial condition of \ref{eq:bvp}.
Then $\alpha - \lambda_{D}(0) \frac{D(0)}{N} = \alpha > 0$.
Following (\ref{eq:opt_U}), the optimal $U(0)=0$.

At the end of the duration, $\lambda_{2}(T)=0$,
which is the boundary condition of \ref{eq:bvp}.
Then $\alpha - \lambda_{D}(T) \frac{D(T)}{N} = \alpha > 0$.
Based on (\ref{eq:opt_U}), the optimal $U(T)=0$.
\end{proof}

Based on the differential function $\dot{I}$,
the equilibrium point of $I$ can be obtained from $\dot{I}=0$,
which is $I^{*}=\frac{\lambda N}{\lambda+\rho}$.
When $I(t)<I^{*}$, $I(t)$ will increase
with $t$ and approach to $\frac{\lambda N}{\lambda+\rho}$.
Meanwhile, in this paper $I(0)=0$ at the beginning of time.
We also note that $I(t)$ will not be effected by
any detection policies.

Based on the differential function $\dot{D}$,
the equilibrium point is obtained from $\dot{D}=0$,
which is $M^{*}=\frac{\rho I}{\beta+\frac{1}{N}U}$.
In the situation without detection,
the equilibrium point is
$D^{*}=\frac{\rho I^{*}}{\beta}=\frac{\rho N}{\beta + \rho}$.
In the situation with full detection,
the equilibrium point is
$D^{*}=\frac{\rho I^{*}}{\beta + \frac{1}{N} U_{m}}
=\frac{\rho}{\beta + \frac{1}{N} U_{m}} \frac{\beta N}{\beta+\rho}$.

Since $\alpha$ is the weight of detecting the selfish nodes,
we can assume that if $\alpha$ is enough high, the detection will not perform according to the optimal control strategy.
\begin{lem}\label{lem:alpha}
If $\alpha \ge \alpha_{th}$, the optimal control let the detection stop in the whole duration,
namely $U(t)=0$, $0 \le t \le T$.
\end{lem}

\begin{proof}
Assume that $\rho$, $N$, $\beta$ is given.
Let $W(t) = \lambda_{D}(t)D(t)$.
\begin{equation}
\begin{small}
%\nonumber
\label{eq:W_diff}
\begin{aligned}
W^{'}(t) =& D^{'}(t) \lambda_{D}(t) + D(t) \lambda_{D}^{'}(t)\\
=& \left(\rho I(t)  - \lambda D(t)
- \frac{D(t)}{N} U(t) \right)\lambda_{D}(t) \\
&+ D(t) \left(\lambda_{D}(t) \left( \lambda + \frac{U(t)}{N} \right)
- (1-\alpha) \right)\\
=& \rho \lambda_{D}(t) I(t) - (1-\alpha)D(t).
\end{aligned}
\end{small}
\end{equation}
Based on (\ref{eq:opt_U}),
we can find that the switching time $t$ is determined by
whether $\lambda_{D}(t)D(t) \le \alpha N$.
Since $D(0)=0$ and $\lambda_{D}(T)=0$,
$W(0)=W(T)=0<\alpha N$.

Now we focus on the poles of $W(t)$, namely $t^{*}$,
where $W^{'}(t^{*})=\rho \lambda_{D}(t^{*}) I(t^{*})
- (1-\alpha)D(t^{*})=0$.
Then $D(t^{*}) = \frac{\rho \lambda_{D}(t^{*}) I(t^{*})}{1-\alpha}$.
\begin{equation}
\begin{small}
%\nonumber
\label{eq:W_t_star}
\begin{aligned}
W(t^{*}) &= \lambda_{D}(t^{*}) D(t^{*})
= \frac{\rho I(t^{*}) \lambda_{D}(t^{*})^2}{1-\alpha}.
\end{aligned}
\end{small}
\end{equation}

According to $\dot{\lambda_{D}}$ in (\ref{eq:bvp}),
the equilibrium point of $\lambda_{D}$
is that $\lambda_{D}^{*} = \frac{1-\alpha}{\lambda+\frac{U}{N}}$.
Since $0 \le U \le U_{m}$,
$0 < \frac{1-\alpha}{\lambda+\frac{U_{m}}{N}}
\le \lambda_{D}^{*} \le \frac{1-\alpha}{\lambda}$.
Note $\lambda_{D}(T)=0$.
Based on the phase line in ODE for $\dot{\lambda_{D}}$,
$\lambda_{D}(t)$ decreases with $t$ when
$\lambda_{D}(t) < \lambda_{D}^{*}$.
Conversely, $\lambda_{D}(t)$ increases with $t$
when $\lambda_{D}(t) > \lambda_{D}^{*}$.
Thus $0 \le \lambda_{D}(t) \le \lambda_{D}^{*}
\le \frac{1-\alpha}{\lambda}$ when $0 \le t \le T$.
Additionally, $0 \le I(t) \le \frac{\lambda N}{\lambda + \rho}$.
From (\ref{eq:W_t_star}), we can derive that
the upper boundary of $W(t)$, $W_{up}$, which is
\begin{equation}
\begin{small}
\nonumber
\begin{aligned}
W(t) \le W(t^{*}) \le \frac{\rho}{1-\alpha}
\frac{\lambda N}{\lambda + \rho} (\frac{1-\alpha}{\lambda})^2
= \frac{\rho N (1-\alpha)}{\lambda(\lambda+\rho)} = W_{up}.
\end{aligned}
\end{small}
\end{equation}

Assume that $\alpha$ can satisfy that $W_{up} \le \alpha N$,
which means that
$\alpha \ge
\frac{\rho}{\lambda(\lambda+\rho)+\rho}
= \alpha_{th}$.
Then $W(t) \le \alpha N$, when $0 \le t \le T$.
Therefore the optimal control $U^{*}(t) \equiv 0$,
when $0 \le t \le T$ in this situation.
\end{proof}

The whole algorithm is shown in Alg.~\ref{alg:opt_detection},
where $\delta t$ presents the time granularity.
First the states of all the relay nodes are set as state $R$.
When the node contacts $src$,
the message will be replicated to the node
and the corresponding state is set as state $I$.
With the time increasing,
the node may discard the message
and then its state may converts to state $D$ at the same time.
According to the results from (\ref{eq:bvp}),
the detection will be conducted at the switching-on duration.
If $src$ detect the node in state $R$ or state $I$,
state will not change.
If the node in state $D$ is selected to detected
its state will be reset as state $I$
and the switching time is recorded by $src$.
Because the message replication time and
the state switching time are recorded by $src$,
the reward of each node,
i.e., the time of the node staying in state $I$ and state $D$,
can also be computed by $src$.
