\section{Related Work}
\label{sec:related}
\subsection{Selfish OppNets}
Observing the importance of OppNets over traditional networks for exchanging information, much efforts has been made to explore the OppNets with selfish nodes in the past few years. 
The majority of existing studies focus on selfish node detection(or misbehavior detection), for the behavior of selfish nodes may cause vulnerability and decrease the performance of OppNets, which was verified and numerical results were given in ~\cite{DBLP:journals/icl/Karaliopoulos09}.
Most selfish node detection approaches can be broadly classified into two groups-the first groups relies on watchdog systems and the other depends on social trust-based communications~\cite{DBLP:journals/comsur/JedariXN18}. 

For example, ~\cite{DBLP:conf/mswim/Hernandez-OralloOCCM12}~\cite{DBLP:journals/tmc/Hernandez-Orallo15} both proposed a collaborative watchdog approach based on the diffusion of selfish nodes awareness.
~\cite{DBLP:journals/fgcs/JedariXCDTA19} proposed a social-based watchdog system(SoWatch) for selfish nodes detection. Compared to most of existing detection schemes that primariliy rely on the nodes' contact records, SoWatch takes nodes' individual and social preferences into account.
Zhu et al. proposed a probabilistic misbehavior detection scheme(iTrust) to judge a node's behavior, based on the collected routing evidences and probabilistically checking in ~\cite{DBLP:journals/tpds/ZhuDGDC14}. 
A metric of misbehavior(MoM) for mathematically evaluating the extent of misbehavior of a node was introduced in ~\cite{DBLP:conf/icc/ChatterjeeSM15}, in which the misbehaving nodes were considered as the voting alternatives and the normally behaving nodes as the voters based on the Theory of Social Choice.
A "Friendship and Selfishness Forwarding"(FSF) algorithm for accessing the relay node's selfishness was presented in ~\cite{DBLP:journals/corr/SouzaMGSMCC17}, with the consideration of the friendship strength among a pair of nodes by using a machine learning algorithm.
~\cite{DBLP:journals/tdsc/ChoC18} proposed a provenance-based trust framework that aims to achieve accurate peer-to-peer trust assessment.
Except approaches mentioned above that belong to the two classes, researchers also investigated other methods to improve selfish node detection.
Basu et al. combined watchdog technique with trust-based communications and integrated with PRoPHET to build a global perception of forwarding behavior for detection of selfish nodes in ~\cite{DBLP:journals/jnca/BasuBRB18}.
Devi V. et al. introduced Semi Markov process for quantifying and future forecasting the probability with which the node could turn into selfish in WSN in~\cite{DBLP:journals/cybersec/VR19}.

Routing is a critical bottleneck after selfish nodes are detected and many literatures designed their routing algorithms for selfish OppNets with incentive mechanisms~\cite{DBLP:journals/access/WangWGFL18}~\cite{DBLP:conf/ciss/ChhabraVS17}~\cite{DBLP:conf/wcsp/LiQZC16}~\cite{DBLP:journals/tvt/CaiFW16}~\cite{DBLP:journals/tpds/ChenSY15}. 
For instance, Li et al. proposed an incentive aware routing for selfish OppNets from a game theoretic perspective, which jointly considered individual selfishness and social selfishness to improve the performance of OppNets in ~\cite{DBLP:conf/wcsp/LiQZC16}.
What's more, Energy-aware routing schemes were presented in ~\cite{DBLP:journals/monet/MaoZ15}~\cite{DBLP:conf/globecom/WuZLYP16}. Mao et al. investigated the energy-aware routing problem in MANETs with nodes' selfishness and solution based on game theory was given in~\cite{DBLP:journals/monet/MaoZ15}. Wu et al. proposed an energy-efficient copy-limit-optimized algorithm for epidemic routing in multi-community scenarios with social selfishness considerations using the Ordinary Differential Equations(ODEs) in ~\cite{DBLP:conf/globecom/WuZLYP16}. ~\cite{7914197} proposed a routing algorithm based on Geographic Information and Node Selfishness, which combines nodes' willingness to forward and their geographic information to maximize the possibility of contacting the destination.

\subsection{Optimizations for Selfish OppNets}
Optimization schemes for selfish OppNets can be classified into several types, the most typical one tries to explore how to control the transmitting process to get a trade-off between the energy consumption and the transmission performance. Since message in selfish OppNets is often transmitted in the store/carry/forward mode that includes beaconing and forwarding process, the optimal control of the message transmission in OppNets contains two parts. 
For instance, Yang et al. proposed optimal energy-efficient neighbor discovery schemes in OppNets in ~\cite{DBLP:conf/ccnc/YangSKK10}~\cite{DBLP:conf/percom/YangSKK09}~\cite{DBLP:journals/jcn/YangSKK15}.
 ~\cite{DBLP:conf/mobicom/PicuS10} applied Markov Chain Monte Carlo (MCMC) optimization to solve the problem of optimal relay selection for group communication, based on node contact patterns. 
After neighborhoods are found in beaconing process, forwarding process transmits the message. 
An optimal replication algorithm(QCR) for impatient content requesters was developed in ~\cite{DBLP:conf/conext/ReichC09}, which drives the global cache in OppNets towards the optimal allocation.
~\cite{DBLP:conf/aina/IppischSG18}introduced a model(ORBOPH) aims for specifying the message copy rule to optimize the distribution of message in OppNets.
 ~\cite{DBLP:journals/jaihc/BorahDWKB18} proposed a multi-objectives based technique for optimized routing(MOTOR) in OppNets, which involves the use of a weighted function to decide on the next hop selection of a node based on a combination of objectives. 
~\cite{DBLP:journals/winet/SharmaRVKC20} modeled the OppIoT environment as a Markov decision process(MDP) and proposed a routing protocol RLProph for routing process optimization, which seeks to fully automate the OppIoT routing process by using the Policy Iteration algorithm to maximize the possibility of message delivery.
~\cite{DBLP:journals/tsipn/ShaghaghianC15} proposed both centralized and decentralized single-copy message forwarding algorithms to minimize the expected latencies from any node in the Opportunistic DTNs.
All of the above works just consider one part of the optimal control of the message transmission in OppNets, Wu et al. considered the optimal forwarding and beaconing control problem at the same time in DTN and gave solution based on Pontryagin's Maximum Principle in ~\cite{DBLP:journals/tcss/WuDH18}, where multiple destinations exist.
