\section{Related Work}
\label{sec:related}
OppNets, which face two challenges,
i.e., energy efficiency and network management cost minimization
are expected to accommodate participants
with low-delay and cost-effective services.
Therefore, many research works targeted to address these issues.

\subsection{Message Transmission in OppNets}
The primary goal of message forwarding in OppNets
is to exploit the nodes' contact, context,
and social information to improve
the message delivery performance
in terms of different metrics(e.g.,
delivery ratio, delay, overhead,
, energy consumption and privacy).
Geo-casting routing protocols, like LoSeRo
\cite{DBLP:journals/tmc/CostantinoMMS20},
exploited knowledge of the most frequently visited
to route message in OppNets, achieved a significant
high delivery rate but
introduced privacy concerns.
Then onion-based anonymous routing approach~\cite{DBLP:conf/icdcs/SakaiSKWA16}
and ePRIVO~\cite{DBLP:journals/tvt/MagaiaBPC18} were proposed to keep users' information private.
To achieve the tradeoff between the delivery rate
and forwarding cost,
game theory was introduced to
optimize the network configuration in MANETs
for more efficient energy-aware routing in~\cite{DBLP:journals/monet/MaoZ15}.
For MOSNs, which exhibits a nested core-periphery hierarchy(NCPH),
\cite{DBLP:journals/tvt/Zheng017} presented an up-and down routing protocol and
achieved a better balance
among the data delivery delay, ratio, and cost.
\cite{DBLP:journals/adhoc/RosasGH20} proposed
a context-aware self-adaptive routing protocol
that is able to adapt to different scenarios.

The majority of aforementioned protocols
assume that mobile nodes willingly participate in data delivery. Nevertheless,
rational nodes in OppNets have strategic interactions
and may exhibit selfish behaviors.
In order to mitigate the performance degradation
caused by the selfishness,
much effort has been made to explore
appropriate incentive mechanisms
to encourage selfish nodes to
participate in data relaying.
For example, the tit-for-tat (TFT)-based scheme,
which forces nodes to exchange the same number
of messages during an opportunistic contact,
was introduced to ensure that mobile nodes provide better
forwarding services for cooperation nodes
but avoid selfish behaviors in
\cite{DBLP:journals/tmc/MastronardePXLS16,
DBLP:journals/twc/HsuD17}.
Trust and reputation-based scheme
\cite{DBLP:journals/tvt/ChenLWW16,DBLP:conf/mdm/JethawaM18}
was adopted to build direct trust relationships
or indirect trust recommendations among mobile nodes.
Eventually, better services were provided
for nodes with high reputation
or strong trust relationships.
Credit-based scheme, such as SEIR~\cite{DBLP:conf/ciss/ChhabraVS17},
Multicent~\cite{DBLP:journals/tpds/ChenSY15},
were designed to encourage the node participation
in data relay by employing various forms
of the virtual credit to
reward cooperative nodes
and punishing them for selfish behaviors.
\subsection{Optimizations of OppNets}
Due to the openness of
the wireless communication and the limitations of the
`store-carry-forward' paradigm,
the performance of data delivery in OppNets may be reduced
by abnormal behaviors.
For example,
the selfish behaviors will highly degrade
the efficiency of data offloading
and the malicious behaviors
will disrupt the
normal communications between the nodes and the network
throughput severely~\cite{DBLP:journals/comsur/JedariXN18}.
It is critical to
identify the abnormal node behaviors from
network accurately and promote these nodes to
participation in collaboration and resource sharing.

Observing the importance of selfish behaviour detection
schemes for the OppNets,
a lot of efforts were
put into its design~\cite{DBLP:journals/fgcs/JedariXCDTA19, Zhou2015Incentive, Chen2014Dynamic, DBLP:journals/tdsc/ChoC18, Wang2016A}.
In~\cite{DBLP:journals/fgcs/JedariXCDTA19},
a general altruism model,
called SoWatch,
was utilized to
distinguish individually selfish behaviour and socially selfish behaviour.
It also showed that the individual and social preferences of selfish nodes
may mitigate the cooperation in data relaying.
An incentive-driven and fresh-ness-aware pub/sub content dissemination scheme
was proposed in~\cite{Zhou2015Incentive},
with the objective of maximizing the utility of the content inventory stored in node's buffer.
\cite{Chen2014Dynamic} proposed a dynamic trust management protocol,
in which `healthiness' and
`unselfishness' are considered as two social trust metrics.
A more comprehensive set of performance metrics to characterize QoS (Quality of Service)
in the OppNets
was investigated in~\cite{DBLP:journals/tdsc/ChoC18}.
To guaranty the security requirements of the OppNets,
a credit-based
rewarding scheme was proposed in~\cite{Wang2016A}.

On the other hand,
the protection and defense mechanisms for malicious behaviors have
been demanded by the OppNets.
To deal with the blackhole and greyhole attacks
in DTNs,
\cite{Pham2016Detecting} proposed a statistical-based detection scheme and
demonstrated its high accuracy.
A compositional secured routing algorithm for the DTNs
was proposed in~\cite{Saha2018Design},
where the information list of malicious nodes was delivered by the trusted nodes.
The types and effectiveness of Sybil attacks in OppNets
was introduced by~\cite{Sacha2016Stalk},
with the consideration of various resource and
attack boosting graph faking attempts.
The problem of crisis of confidence caused by malicious nodes
was proposed in~\cite{Yao2016Secure},
and a dynamic trust management model
was presented there to solve the problem.
In~\cite{Chen2016Trust},
an adaptive trust management
protocol for social IoTs (Internet of Things) systems was proposed
to choose the trust parameter settings and change node's social conditions.
%Optimization schemes of OppNets can be classified into
%several types, the most typical one tries to formulate
%the transmitting process in terms of a trade-off between
%the network management cost and the transmission performance.
%For example, on optimal neighbor discovery,
%PWEND~\cite{DBLP:journals/adhoc/ChenQLLWYL20} and
%Pharos~\cite{DBLP:conf/secon/Zhu00L19} adopted time model
%for neighbor discovery and investigated the most energy efficient way
%and the least discovery latency, respectively.
%Then for a given energy budget,
%how to optimizing the number of discovered peers was researched
%in~\cite{DBLP:journals/tmc/LoretiB20},
%what is the best achievable discovery latency was addressed
%by~\cite{DBLP:conf/sigcomm/KindtC19}.
%
%As for optimal data forwarding,
%\cite{DBLP:journals/tvt/ZhouLZXF17} proposed an
%efficient time-aware data forwarding strategy(TCCB) for OMNs,
% based on temporal social contact patterns.
% The model performed a close delivery ratio to
% Epidemic but with significantly reduced delivery cost.
%\cite{DBLP:journals/tvt/LiuWXWLY17} introduced
%a centralized heuristic algorithm
%which aimed to discover a tree for multicasting,
%with resource constrained (i.e. the delay-constrained least-const) in MONs.
%Both centralized and decentralized single-copy message forwarding algorithms
%were proposed in~\cite{DBLP:journals/tsipn/ShaghaghianC15},
%which aimed to minimize the expected latencies
%from any node in the Opportunistic DTNs.
%However, aforementioned works just consider one part of
%the message transmission in OppNets,
%\cite{DBLP:journals/tcss/WuDH18} mathematically characterized
%message transmission of the selfish
%and altruistic cases as an optimal control problem,
%whose controlling parameters were chosen according
%to the forwarding rate and beaconing rate, respectively.
%Then the Pontryagin's Maximum Principle was exploited
%to search the problem solution in multiple destinations scenario
%and the optimal control policies were proved to satisfy the threshold form.
%%Wu et al. considered the optimal forwarding and beaconing control problem at the same time in DTN and gave solution based on Pontryagin's Maximum Principle in ~\cite{DBLP:journals/tcss/WuDH18}, where multiple destinations exist.
%
%Minimize the contact duration by optimizing
%mobile data offloading in OMNs is the objective
%of~\cite{DBLP:journals/tits/LiJWZ014,DBLP:conf/icc/WangW18}.
%A mathematical framework to study the problem of
%coding-based mobile data offloading
%was established in~\cite{DBLP:journals/tits/LiJWZ014},
%the authors formulated the problem as a users' interest
%satisfaction maximization problem
%with multiple linear constraints of
%limited storage and efficient scheme
%was proposed to solve it.
%An optimal traffic offloading scheme through
%data partition, which generated forwarding paths
%with possible heterogeneous data chunks,
%was presented in~\cite{DBLP:conf/icc/WangW18}.

Few these existing works focus on
optimal control policy, while we introduce it
for selfish node detection,
where the scenario is different from~\cite{DBLP:journals/tcss/WuDH18} in this paper.
