\section{Related Work}
\label{sec:related}
OppNets, which face two challenges,
i.e., energy efficiency and network management cost minimization
are expected to accommodate participants
with low-delay and cost-effective services.
Therefore, many research works targeted to address these issues.

\subsection{Message Transmission in OppNets}
In order to mitigate the performance degradation caused by
the selfish behaviours in OppNets,
much effort has been made to explore
the methods of selfish node
detection~\cite{DBLP:journals/tvt/LiSWJSZ11, DBLP:journals/comsur/JedariXN18}.
%To achieve message transmission in selfish OppNets,
%the very first step is to find selfish nodes,
%due to their uncooperative behaviour causes vulnerability
%and decreases the performance
%of OppNets
An early investigation on the selfish behaviour detection
is~\cite{DBLP:conf/mobicom/MartiGLB00},
where the watchdog nodes were proposed
to analyze the traffic received from their encountered nodes.
%The first approach implements the watchdog nodes to analyze the traffic received from their encountered nodes to decide whether they have selfish behaviour in message relaying or not~\cite{DBLP:conf/mobicom/MartiGLB00}.
This work was extended for applications with
the elimination of the limited knowledge on node detection by single watchdog,
and the cooperative systems with multiple watchdogs were proposed
in~\cite{DBLP:journals/tmc/Hernandez-Orallo15, DBLP:journals/tie/DiasRXM15,
DBLP:journals/fgcs/JedariXCDTA19}.
%Nevertheless, poor performance mould incur if relying on local watchdogs only, and a number of cooperative watchdog systems have been proposed to tackle this problem in recent few years~\cite{DBLP:journals/tmc/Hernandez-Orallo15,
%DBLP:journals/tie/DiasRXM15,
%DBLP:journals/fgcs/JedariXCDTA19}.
\cite{DBLP:journals/tmc/Hernandez-Orallo15} proposed
a collaborative approach (CoCoWa, Collaborative Contact-based Watchdog),
which considered the diffusion of local selfish nodes awareness,
to conduct the selfish node detection in MANETs.
Through accelerating the information propagation,
the method improved the performance of selfish node detection
in terms of the time and the precision.
\cite{DBLP:journals/fgcs/JedariXCDTA19} proposed
a social-based watchdog system (SoWatch),
with a watchdog module to protect SoWatch
against the wrong watchdogs manipulated by malicious nodes.

%categories
%Selfish node detection strategies are classified into two main classes:
%watchdog systems and social trust-based communications in the work~
%\cite{DBLP:journals/comsur/JedariXN18}.
Another kind of approach tries to establish social trust relationships
between mobile nodes in OppNets by leveraging their online social information
(explicit trust) as well as their interactions or mobility properties (implicit trust).
In~\cite{DBLP:journals/tpds/ZhuDGDC14}, a probabilistic misbehavior detection scheme(iTrust),
which introduced a periodically available Trusted Authority(TA),
was presented to judge a node's behaviour.
Another trust framework PROVEST (PROVEnance-baSed Trust model)
that aimed to achieve accurate peer-to-peer trust assessment was presented
in~\cite{DBLP:journals/tdsc/ChoC18}.
The partial selfishness was investigated and credit-based algorithm to
measure the degree of selfishness was designed in~\cite{DBLP:journals/tmc/ChoiSLW12}.

\cite{DBLP:journals/jnca/BasuBRB18} combined watchdog technique
with trust-based communications and integrated with PRoPHET
to build a global perception of forwarding behavior for detection of selfish nodes.
\cite{DBLP:conf/icdcs/GaoMAH18} introduced ensemble learning
for environment-adaptive malicious node detection.
\cite{DBLP:journals/tvt/NomikosCVWK20} integrated
buffer-aided full-duplex/half-duplex relaying with
non-orthogonal multiple access(BAHyNOMA) for relay selection.

Routing is a critical bottleneck when selfish behaviour
is exhibited and a potential alleviation is to
develop incentivizing mechanisms for message forwarding.
Incentive-based protocols, such as SEIR~\cite{DBLP:conf/ciss/ChhabraVS17},
Multicent~\cite{DBLP:journals/tpds/ChenSY15},
were devised to increase node participation in message forwarding
by opting for mechanisms that reward active participation of nodes
in the forwarding of messages and penalize them otherwise.
To balance the tradeoff between the delivery rate and forwarding cost,
game theory was introduced to optimize the configuration in MANET
for more efficient energy-aware routing in~\cite{DBLP:journals/monet/MaoZ15}.
While geo-casting routing protocols like LoSeRo~\cite{DBLP:journals/tmc/CostantinoMMS20}
exploited the location data to enhance the message routing performance,
onion-based anonymous routing approach~\cite{DBLP:conf/icdcs/SakaiSKWA16}
and ePRIVO~\cite{DBLP:journals/tvt/MagaiaBPC18} were proposed
to keep users' information private.
For MOSNs, which exhibits a nested core-periphery hierarchy(NCPH),
\cite{DBLP:journals/tvt/Zheng017} presented an up-and down routing protocol
to upload message from source node to the network core
and then download to the destination.
\cite{DBLP:journals/adhoc/RosasGH20} proposed
a context-aware self-adaptive routing protocol
that is able to adapt to different scenarios.

\subsection{Optimizations of OppNets}
Due to the openness of
the wireless communication and the limitations of the
`store-carry-forward' paradigm,
the performance of data delivery in OppNets may be reduced
by abnormal behaviors.
For example,
the selfish behavior will highly degrade
the efficiency of data offloading
and the malicious behaviors
will disrupt the
normal communications between the nodes and the network
throughput severely~\cite{DBLP:journals/comsur/JedariXN18}.
It is critical to
identify the abnormal node behaviors from
network accurately and promote these nodes to
participation in collaboration and resource sharing.

Observing the importance of selfish behaviour detection
schemes for the OppNets,
a lot of efforts were
put into its design~\cite{DBLP:journals/fgcs/JedariXCDTA19, Zhou2015Incentive, Chen2014Dynamic, DBLP:journals/tdsc/ChoC18, Wang2016A}.
In~\cite{DBLP:journals/fgcs/JedariXCDTA19},
a general altruism model,
called SoWatch,
was utilized to 
distinguish individually selfish behaviour and socially selfish behaviour.
It also showed that the individual and social preferences of selfish nodes
may mitigate the cooperation in data relaying.
An incentive-driven and fresh-ness-aware pub/sub content dissemination scheme
was proposed in~\cite{Zhou2015Incentive},
with the objective of maximizing the utility of the content inventory stored in node's buffer.
\cite{Chen2014Dynamic} proposed a dynamic trust management protocol,
in which `healthiness' and
`unselfishness' are considered as two social trust metrics. 
A more comprehensive set of performance metrics to characterize QoS in the OppNets
was investigated in~\cite{DBLP:journals/tdsc/ChoC18}.
To guaranty the security requirements of the OppNets,
a credit-based
rewarding scheme was proposed in~\cite{Wang2016A}.

On the other hand, 
the protection and defense mechanisms for malicious behaviors have
been demanded by the OppNets.
To deal with the blackhole and greyhole attacks
in DTN,
\cite{Pham2016Detecting} proposed a statistical-based detection scheme and
demonstrated its high accuracy. 
A compositional secured routing algorithm for the DTNs
was proposed in~\cite{Saha2018Design},
where the information list of malicious nodes was delivered by the trusted nodes.


Optimization schemes of OppNets can be classified into
several types, the most typical one tries to formulate
the transmitting process in terms of a trade-off between
the network management cost and the transmission performance.
For example, on optimal neighbor discovery,
PWEND~\cite{DBLP:journals/adhoc/ChenQLLWYL20} and
Pharos~\cite{DBLP:conf/secon/Zhu00L19} adopted time model
for neighbor discovery and investigated the most energy efficient way
and the least discovery latency, respectively.
Then for a given energy budget,
how to optimizing the number of discovered peers was researched
in~\cite{DBLP:journals/tmc/LoretiB20},
what is the best achievable discovery latency was addressed
by~\cite{DBLP:conf/sigcomm/KindtC19}.

As for optimal data forwarding,
\cite{DBLP:journals/tvt/ZhouLZXF17} proposed an
efficient time-aware data forwarding strategy(TCCB) for OMNs,
 based on temporal social contact patterns.
 The model performed a close delivery ratio to
 Epidemic but with significantly reduced delivery cost.
\cite{DBLP:journals/tvt/LiuWXWLY17} introduced
a centralized heuristic algorithm
which aimed to discover a tree for multicasting,
with resource constrained (i.e. the delay-constrained least-const) in MONs.
Both centralized and decentralized single-copy message forwarding algorithms
were proposed in~\cite{DBLP:journals/tsipn/ShaghaghianC15},
which aimed to minimize the expected latencies
from any node in the Opportunistic DTNs.
However, aforementioned works just consider one part of
the message transmission in OppNets,
\cite{DBLP:journals/tcss/WuDH18} mathematically characterized
message transmission of the selfish
and altruistic cases as an optimal control problem,
whose controlling parameters were chosen according
to the forwarding rate and beaconing rate, respectively.
Then the Pontryagin's Maximum Principle was exploited
to search the problem solution in multiple destinations scenario
and the optimal control policies were proved to satisfy the threshold form.
%Wu et al. considered the optimal forwarding and beaconing control problem at the same time in DTN and gave solution based on Pontryagin's Maximum Principle in ~\cite{DBLP:journals/tcss/WuDH18}, where multiple destinations exist.

Minimize the contact duration by optimizing
mobile data offloading in OMNs is the objective
of~\cite{DBLP:journals/tits/LiJWZ014,DBLP:conf/icc/WangW18}.
A mathematical framework to study the problem of
coding-based mobile data offloading
was established in~\cite{DBLP:journals/tits/LiJWZ014},
the authors formulated the problem as a users' interest
satisfaction maximization problem
with multiple linear constraints of
limited storage and efficient scheme
was proposed to solve it.
An optimal traffic offloading scheme through
data partition, which generated forwarding paths
with possible heterogeneous data chunks,
was presented in~\cite{DBLP:conf/icc/WangW18}.
\cite{Sacha2016Stalk}

Few these existing works focus on
optimal control policy, while we introduce it
for selfish node detection,
where the scenario is different from~\cite{DBLP:journals/tcss/WuDH18} in this paper.
